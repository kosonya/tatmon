% LaTeX file for resume 
% This file uses the resume document class (res.cls)

\documentclass{res} 
%\usepackage{helvetica} % uses helvetica postscript font (download helvetica.sty)
%\usepackage{newcent}   % uses new century schoolbook postscript font 
\setlength{\textheight}{9.5in} % increase text height to fit on 1-page 

\usepackage{hyperref}
\usepackage[T2A,T1]{fontenc}
\usepackage[utf8]{inputenc}
\usepackage[english,russian]{babel}

\begin{document} 

\name{MAXIM KOVALEV\\[12pt]}     % the \\[12pt] adds a blank
				        % line after name      

\address{\bf  PRESENT ADDRESS\\201 Thorn Lane\\apt \#10\\Newark, DE 19711}
\address{\bf PERMANENT ADDRESS \\ 14/5 Pokrovsky boulevard \\apt \#25\\Moscow, Russia 109028}
                         

         
\begin{resume}
Cell phone: +1 (302) 415-4176; +7 (903) 245-2989.\\
Email: \href{mailto:maxikov@udel.edu}{maxikov@udel.edu}.\\
Skype: mmkovalev.\\

\section{OBJECTIVE}          
To earn a Ph.D. degree in the field of the machine learning at Stanford and to receive a funding for my studies.          
 
\section{EDUCATION}
\begin{itemize}
	\item
		Moscow State Institute of Electronics and Mathematics\\
		Engineer degree in Computational Machines, Complexes, Systems and networks, June 2012\\
		(equivalent to the Bachelor degree in Computer Science)\\
		G.P.A. 4.9 out of 5.0\\
		Graduate Thesis: ``Development of the public transportation vehicle tracking system based on GSM signal''
	\item
		English Language Institute of the University of Delaware\\
		English language practice, Present
\end{itemize}

 
\section{EXPERIENCE}
\vspace{-0.1in}	
\begin{tabbing}
\hspace{1.3in}\= \hspace{3.6in}\= \kill % set up two tab positions
{\bf Technician / TA} \>Moscow State Institute of Electronics and Mathematics \>Fall 2010 -- Summer 2012\\
			\>Moscow, Russia \> (started unofficially in 2008)
\end{tabbing}      % suppress blank line after tabbing
\begin{itemize}
	\item
		Preparing synopsis and other didactic materials for course ``Programming in the high-level programming languages'' in MediaWiki-based informational system.
	\item
		Preparing tests and grading quizzes for ``Programming\dots'' course in the Moodle-based learning management system.
	\item
		Building the syllabus for ``Programming\dots'' course from scratch while switching from C to Python as a first language.
	\item
		Teaching nearly all ``Programming\dots'' classes on the last year.
	\item
		Teaching the part about \LaTeX{} in the ``Information science'' course.
\end{itemize}

\section{COMPUTER SKILLS}          
\begin{itemize}
	\item
		Programming in C/C++ and Python.
	\item
		Using Eclipse IDE and Subversion.
	\item
		Typesetting academic papers in the \LaTeX{} system.
	\item
		General GNU/Linux (especially Ubuntu and Gentoo) usage and administration (including Netfilter and KVM).
	\item
		Minor experience in GNU/Octave, Maple, SQL, HTML, Haskell, Android SDK (Java).
\end{itemize}

\section{LANGUAGES SPOKEN}
\begin{enumerate}
	\item
		Russian (native);
	\item
		English;
	\item
		Japanese (JLPT level N4);
	\item
		French (basic).
\end{enumerate}
 
\section{PUBLICATIONS}
\begin{itemize}
	\item
		Kovalev, M. (2012). Prototype of the software and hardware complex for collecting data about GSM signal levels. \emph{Proceedings of the annual scientific and technical conference of undergraduate and graduate students, and young experts of Moscow State Institute of Electronics and Mathematics} (pp. 185--187). Moscow, Russia: MIEM Press. ISBN 978-5-94506-314-3.\\
		(М.М.Ковалев. Прототип программно-аппаратного комплекса для сбора данных об уровнях сигнала GSM на местности. Научно-техническая конференция студентов, аспирантов и молодых специалистов МИЭМ, посвящённая 50-летию МИЭМ. Тезисы докладов. -- М.: МИЭМ, 2012. --- 445 с. ISBN 978-5-94506-314-3. с.с. 185--187.)
	\item
		Kovalev, M. (2011). Positioning of the surface transport vehicles by means of GSM signal. \emph{Proceedings of the annual scientific and technical conference of undergraduate and graduate students, and young experts of Moscow State Institute of Electronics and Mathematics} (pp. 29--30). Moscow, Russia: MIEM Press. ISBN 978-5-94506-257-3.\\
		(М.М.Ковалев. Позиционирование наземного транспорта с помощью сигнала GSM. Научно-техническая конференция студентов, аспирантов и молодых специалистов МИЭМ. Тезисы докладов. -- М.: МИЭМ, 2011. --- 420. ISBN 978-5-94506-257-3. с.с. 29--30.)
	\item
		Kovalev, M. (2008). Cyclic development of Russia \emph{Proceedings of the annual scientific and technical conference of undergraduate and graduate students, and young experts of Moscow State Institute of Electronics and Mathematics} (pp. 398--399). Moscow, Russia: MIEM Press. ISBN 978-5-94506-170-5.\\
		(М.М.Ковалев. Циклическое развитие России. Научно-техническая конференция студентов, аспирантов и молодых специалистов МИЭМ. Тезисы докладов. -- М.: МИЭМ, 2008. --- 495. ISBN 978-5-94506-170-5. с.с. 398--399.)

\end{itemize}

\section{HONORS AND AWARDS}          
\begin{itemize}
	\item
		Commendation for the active participation in implementation of outside computational practice in the ``Ruza'' camp of MIEM \emph{(by V.N. Azarov, vice-chancellor of scientific work and informatization in MIEM, 2010)}.
	\item
		Level I diploma for the best work presented on scientific and technological conference of undergraduate and graduate students, and young experts of MIEM \emph{(by V. Bykov, chancellor of Moscow State Institute of Electronics and Mathematics, 2008)}.
	\item
		Award for finalist of the contest of video presentations of projects for the project ``Tracking of surface transport vehicles based on GSM signal'' \emph{(by organizing committee of the ``Telecom Idea'' contest, 2011)}.
	\item
		Certificate of honor for the winner of ``Participant of Youth Scientific and Innovative Contest'' program \emph{(by I. Bortnik, chairman of surveillance council and S. Polyakov, CEO of foundation for assistance to small scientific and technological business, 2011)}.
	\item
		Diploma for the winer of ``Participant of Youth Scientific and Innovative Contest'' program of 2011 year for the work presented on scientific an technological conference of undergraduate and graduate students, and young experts of MIEM \emph{(by V. Kulagin, acting chancellor of Moscow State Institute of Electronics and Mathematics, 2011)}.
	\item
		Certificate of honor for the success in educational and scientific activities and for the 50-year anniversary of MIEM \emph{(by V. Kulagin, chancellor of Moscow State Institute of Electronics and Mathematics, 2012)}.
\end{itemize}


\section{MISCELLANEOUS}          
 
Received a grant for my graduate thesis research in 2011.\\
Studied Japanese for 2 years in the Autonomous Non-commercial Association ``The Japan center'' in Moscow State University.\\
Attended lectures of the ``Fundamentals of neurobiology'' open class in Moscow State University in 2009.\\
Completed the online Machine Learning course by A. Ng in 2011. Score on review questions: 59 out of 80; score on programming exercises: 750 out ot 800.\\
Completed the free online offering of Introduction to Databases by J. Widom in 2011. Scaled total score: 280.5 out of 323

\end{resume}
\end{document}
