%This file is part of diplom-kovalev
%Copyright (C) 2012 Maxim Kovalev.
%Permission is granted to copy, distribute and/or modify this document
%under the terms of the GNU Free Documentation License, Version 1.3
%or any later version published by the Free Software Foundation;
%with no Invariant Sections, no Front-Cover Texts, and no Back-Cover Texts.
%A copy of the license is included in the section entitled "GNU
%Free Documentation License".

\chapter*{Аннотация}
\addcontentsline{toc}{chapter}{Аннотация}

В данной работе создан и проверен новый метод позиционирования мобильных устройств на основе информации о принимаемых ими уровнях сигналов от базовых станций сетей GSM, рассчитанный на стационарную установку позиционируемых устройств на наземном общественном транспорте. Использование априорной информации о неизменном маршруте движения каждого транспортного средства позволило свести задачу от двухмерной (поиск широты и долготы) к одномерной (поиск расстояния от фиксированной точки на маршруте) и полностью отказаться от применения метода триангуляции в пользу статистического подхода.

Разработанный алгоритм был реализован в прототипе системы позиционирования, после чего проверен на тестовом отрезке пути в реальных городских условиях. Достигнута средняя погрешность определения положения в 49 метров, что лучше, чем у современных систем, основанных на триангуляции, но ещё не позволяет сравниться с системами спутниковой навигации, а потому для достижения возможности внедрения наравне со спутниковыми системами, требуется дополнительная научная и техническая работа.

Исходные коды компонентов разработанной системы, текстовые журналы экспериментов, а также исходные коды данной пояснительной записки в форматах \LaTeX{} и Graphviz вместе с использованными растровыми изображениями доступны в репозитории Subversion по адресу http://svn.auditory.ru/repos/tatmon.

В соответствие с GNU Free Documentation License\cite{gnufdl}, данная работа является свободным документом:
\bigskip
\begin{quote}
    Copyright \copyright{}  2012  Maxim Kovalev.
    Permission is granted to copy, distribute and/or modify this document
    under the terms of the GNU Free Documentation License, Version 1.3
    or any later version published by the Free Software Foundation;
    with no Invariant Sections, no Front-Cover Texts, and no Back-Cover Texts.
\end{quote}
\bigskip

