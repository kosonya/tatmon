%This file is part of diplom-kovalev
%Copyright (C) 2012 Maxim Kovalev.
%Permission is granted to copy, distribute and/or modify this document
%under the terms of the GNU Free Documentation License, Version 1.3
%or any later version published by the Free Software Foundation;
%with no Invariant Sections, no Front-Cover Texts, and no Back-Cover Texts.
%A copy of the license is included in the section entitled "GNU
%Free Documentation License".

\chapter*{Заключение}
\addcontentsline{toc}{chapter}{Заключение}

В данной работе был предложен и опробован принципиально новый метод позиционирования устройств на основе сигналов базовых станций сетей GSM, рассчитанный на использование в системах наблюдения за наземным общественным транспортом. Качество работы метода превысило то, которое дают существующие, основанные на триангуляции в сетях GSM, но ещё не достигло точности, сравнимой со спутниковыми системами, а потому требуются и были предложены дополнительные экспериментальные и теоретические изыскания, направленные на дальнейшее совершенствование метода.

В обзорно-аналитической части были рассмотрены существующие методы позиционирования устройств: спутниковая навигация и триангуляция в сетях GSM. Были подробно исследованы особенности метода триангуляции и его практического применения, был сделан вывод о роли городской застройки как, предположительно, главного фактора, ограничивающего точность подобного метода. Было отмечено, что особенности общественного транспорта, как движущегося всегда по ограниченному и заранее известному набору маршрутов, позволяют отказаться от метода триангуляции, с присущими ему недостатками, в пользу статистических методов. Сформулирована гипотеза о том, что такой переход позволит значительно повысить точность позиционирования, возможно, вплоть до значений, обеспечиваемых спутниковой навигацией.

После этого был осуществлён анализ существующих методов поиска наибольшего правдоподобия аргумента при сравнении векторов случайных переменных, таких как расстояние Махаланобиса и Байесовский классификатор, в результате которого сделан вывод о том, что ни один из них в полной мере не удовлетворяет требованиям, предъявляемым к алгоритму позиционирования транспорта, хотя каждый из них обладает своими достоинствами, в результате чего сделан вывод о необходимости создания нового математического метода.

Создание требуемого алгоритма является одним из ключевых достижений данной работы. Была предложена функция псевдоплотности вероятности, позволяющая создать алгоритм, напоминающий Байесовский классификатор и обладающий всеми его достоинствами, такими как большая устойчивость к выбросам и шумам в значениях анализируемых случайных величин, но работающий с бесконечными, причём континуальными и упорядоченными множествами как возможных значений случайных величин, так и классов. Из возможности работы над континуальными множествами следует то, что алгоритм, во-первых, для каждой возможной точки учитывает не только данные о ней самой, но и данные о соседних точках, во-вторых, пользуется не дискретным сравнением совпадения данного ответа с хранящимся в базе, а использует нечёткую шкалу похожести, а в-третьих, позволяет успешно обрабатывать ситуацию, когда даётся ответ, никогда ранее не дававшийся, но похожий на имеющиеся.

Этот алгоритм был положен в основу прототипа программно-аппаратного комплекса для позиционирования, состоящего из четырёх звеньев:

\begin{enumerate}
	\item
		Мобильное устройство, собирающее данные об уровнях сигналов видимых базовых станций GSM в разных точках маршрута и передающее их на сервер; в качестве устройства использовался смартфон со специальной программой, написанной на языке Java и работающей под управлением ОС Android;
	\item
		Сервер, принимающий и сохраняющий данные об уровнях сигнала, а затем использующий их в работе созданного алгоритма позиционирования; написан на языке Python с использованием библиотеки NumPy для векторных вычислений;
	\item
		База данных, хранящая информацию об уровнях сигнала; использована СУБД MySQL;
	\item
		Клиентское приложение, получающее информацию о процессе сбора данных или позиционирования с сервера в реальном времени и отображающее её на карте местности; написано на языке Python с использованием библиотеки pygame для отображения графики.
\end{enumerate}

После завершения разработки системы была предложена методика проведения экспериментальной проверки гипотезы о возможности повышения точности позиционирования с использованием данной системы. Проверка осуществлялась на прямом тестовом участке на Покровском и Яузском бульварах в Москве. В ходе предварительного сбора данных было осуществлено боле 20 тысяч замеров уровней сигналов базовых станций в разных точках, при этом в разное время в зоне видимости были видны 22 различные базовые станции. Когда данные были собраны, было поставлено 176 экспериментов по позиционированию устройства, пользуясь собранными данными и разработанным алгоритмом, контроль при этом осуществлялся по данным GPS.

В результате, математическое ожидание ошибки позиционирования составило 49 метров, что лучше современных систем, основанных на триангуляционном методе, но ещё не достигает точности спутниковой навигации. На основании интерпретации экспериментальных данных, в том числе и трассировки процесса позиционирования на реальных данных, были выдвинуты гипотезы о факторах, которые могут мешать дальнейшему повышению точности, и предложены эксперименты для выяснения, возможно ли дальнейшее повышение точности с помощью статистических методов, или же был достигнут технологический предел точности позиционирования на основе данных об уровнях сигналов базовых станций.

{\bf{}Вывод} из данной работы можно сформулировать следующим образом: созданный математический метод показал большую точность позиционирования, чем триангуляция, и гипотеза о возможности повышения точности, используя статистический подход, оправдалась. При этом, достигнуть той точности, которую даёт спутниковая навигация, пока не удалось, но были предложены пути для дальнейшего совершенствования метода.
